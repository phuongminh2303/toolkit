\chapter{BÀN LUẬN}

\textbf{
\section{Đặc điểm màu sắc răng theo bảng màu Vita 3D Master}}

\qquad Biểu đồ 3.1.1 cho thấy tỷ lệ người bệnh nữ cao hơn so với người bệnh nam 2,125 lần.\par
\quad Độ tuổi của đối tượng nghiên cứu từ 20-24 tuổi, là sinh viên trường Đại học Y Dược - Đại học Quốc gia Hà Nội.\par
\quad Bảng 3.1.1 cho thấy tiền sử sử dụng Tetracycline của các đối tượng khá thấp (21,5\% tổng số) và ở nữ nhiều hơn ở nam giới, (nữ: 15,4\% và nam: 6,2\%). Kết quả ở nghiên cứu của Nguyễn Thị Châu \cite{NguyenThiChau} cũng tương tự khi tỷ lệ người bệnh nữ từng sử dụng Tetracycline cao hơn nhiều so với nam. Tình trạng người bệnh 20-24 tuổi có tiền sử sử dụng Tetracycline tương đối thấp so với nhóm tuổi trên 30\cite{NguyenThiChau}.\par
\quad Bảng 3.1.2 cho thấy thói quen sử dụng đồ có màu ở cả hai giới không có sự khác biệt nhiều. Tỷ lệ thói quen sử dụng cocacola là cao nhất (60\%), uống cà phê chiếm 46,2\%, và đồ uống có màu khác chiếm khoảng 44,6\%. Hút thuốc lá chiếm 9,2\% và chỉ gặp ở nam giới, không gặp ở nữ giới. Kết quả này tương đồng với nghiên cứu của Nguyễn Thị Châu\cite{NguyenThiChau}, cho thấy thói quen hút thuốc lá thường gặp ở nam giới. Đặc điểm thói quen sử dụng thực phẩm có màu ảnh hưởng nhiều tới đặc điểm màu sắc răng.\par
\quad Bảng 3.1.3 cho thấy tình trạng vệ sinh răng miệng ở hai giới tương đối đồng đều. Toàn bộ sinh viên vệ sinh răng miệng $\ge$ 2 lần/ngày. Theo báo cáo của Trịnh Thị Tố Quyên\cite{trinhthitoquyen}, 87,8\% đối tượng chải răng hai lần hoặc trên hai lần mỗi ngày và ở báo cáo của Trịnh Minh Báu ở 770 sinh viên, tỷ lệ đánh răng $\ge$ 2 lần/ngày là 90,1\%\cite{trinhminhbau}. Điều này cho thấy ý thức vệ sinh răng miệng hiện nay của sinh viên các trường đại học y đều khá cao và rất tốt.\par
\quad Tỷ lệ sử dụng nước súc miệng có màu sẫm tương đối thấp, khoảng 27,7\%. Tỷ lệ sử dụng nước súc miệng của sinh viên theo Trịnh Minh Báu lại khá cao, 72,2\%.\cite{trinhminhbau}\par
\quad Tình trạng chăm sóc răng miệng định kỳ 6 tháng 1 lần chiếm tỷ lệ khoảng 46,2\%. Tỷ lệ trên 1 năm đi lấy cao răng 1 lần chiếm khoảng 29,2\%. Chưa đi lấy cao răng bao giờ chiếm tỷ lệ 18,5\%. Những con số này cho thấy giới trẻ ngày nay có ý thức cao hơn trong việc chăm sóc vệ sinh răng miệng. Tuy nhiên, thói quen đi kiểm tra định kỳ răng miệng vẫn chưa phổ biến rộng rãi.\par 
\quad Bảng 3.1.4 cho thấy tỷ lệ bề mặt răng nhẵn bóng (92,3\%) cao hơn so với bề mặt răng sần sùi (7,7\%). Đa số các răng không rạn nứt (98,5\%). Không có răng sâu nào. Tỷ lệ tính chất màu sắc răng đồng nhất là 87,7\%, cao hơn tỷ lệ tính chất màu sắc răng không đồng nhất khoảng 7 lần (12,3\%). Ở độ tuổi 20-24, tỷ lệ cao của bề mặt răng nhẵn bóng, răng không bị rạn nứt và không có răng sâu cho thấy tình trạng răng miệng khá tốt. Tuy nhiên, vẫn còn có sự khác biệt về tình trạng răng, nhưng tỷ lệ này không lớn.\par
\quad Hiện nay có nhiều phương pháp đánh giá màu răng: sử dụng bảng so màu bằng băng giấy, sứ có màu hoặc nhựa acrylic, sử dụng phổ quang kế, sắc kế và kỹ thuật phân tích hình ảnh theo Joiner. Nghiên cứu này sử dụng bảng so màu Vita 3D Master.\par
\quad Màu sắc theo bảng Vita 3D Master xác định màu theo phổ màu Munsell: Chroma (độ bão hoà màu), Hue (tông màu), Value (giá trị màu). Phương pháp trực quan này mang tính chủ quan phụ thuộc vào người quan sát và các yếu tố ảnh hưởng tới sự so màu: ánh sáng, thời gian trong ngày, điều kiện thời tiết, môi trường xung quanh, các yếu tố liên quan đến tuổi, kinh nghiệm làm việc, mệt mỏi và trạng thái cảm xúc.\par
\quad Bảng 3.1.5 cho thấy tỷ lệ phân bố giá trị màu sắc răng ở nhóm 3 là cao nhất ở cả 3 nhóm răng (nhóm răng cửa (50,8\%), nhóm răng nanh (53,8\%), nhóm răng hàm nhỏ(53,8\%)) và tương đối ngang nhau giữa hai giới nam và nữ.\par 
\quad Tỷ lệ giá trị màu sắc răng ở nhóm 2 xếp sau với nhóm răng cửa khoảng 43\%, nhóm răng nanh khoảng 40\% và nhóm răng hàm nhỏ khoảng 41,5\%.\par
\quad Nhóm 1 chỉ chiếm tỷ lệ rất nhỏ là 1,5\% và không có đối tượng nào có màu sắc theo nhóm 5.\par
\quad Theo nghiên cứu của Trần Nguyên Lộc\cite{trannguyenloc}, theo bảng màu V3D-M, màu răng có tần suất cao nhất là 2M2 (21,9\%), kế đến là 2M3 (21,1\%). Màu sắc của nhóm nghiên cứu này có tỷ lệ màu răng thuộc nhóm 2 là cao nhất, sự khác biệt kết quả này có thể do rất nhiều nguyên nhân và yếu tố bên ngoài như thói quen sử dụng đồ uống có màu, vệ sinh răng miệng và thói quen khám răng định kỳ.\par 
\quad Bảng 3.1.6 cho thấy tỷ lệ phân bố giá trị màu sắc răng ở nhóm 3 là cao nhất ở cả 3 nhóm răng (nhóm răng cửa (68,8\%), nhóm răng nanh (61\%), nhóm răng hàm nhỏ (70,3\%)) và tương đối ngang nhau giữa hai giới nam và nữ.\par
\quad Tỷ lệ giá trị màu sắc răng ở nhóm 2 xếp sau với nhóm răng cửa khoảng 26,6\%, nhóm răng nanh khoảng 25\% và nhóm răng hàm nhỏ khoảng 25\%.\par
\quad Nhóm 1 và nhóm 5 không có đối tượng nào. \par
\quad Biểu đồ 3.2.2 cho thấy tỷ lệ màu sắc răng cao nhất ở các đối tượng là màu 3M2, tiếp theo là 3M3 và 3R2.5.\par
\quad Tỷ lệ màu sắc răng thấp nhất là màu 4M1,4R2.5 và 1M2.
So sánh với kết quả của Trần Nguyên Lộc\cite{trannguyenloc} thấy được đặc điểm màu sắc răng có sự chênh lệch về màu sắc trên lâm sàng, nguyên nhân có thể đến từ sự khác biệt về độ sáng tối và tông màu, có thể do nguyên nhân ngoại sinh như sử dụng thực phẩm có màu và thói quen vệ sinh răng, tái khám định kỳ là chưa tốt.




\textbf{
\section{Đánh giá nhu cầu làm trắng răng}}
\quad Theo nghiên cứu của Renata Pedrosa Guimarães (2021)\cite{Renata2021}, các nhận thức về những gì là thẩm mỹ hoặc không thẩm mỹ, liên quan đến màu sắc của răng, thay đổi tùy thuộc vào bối cảnh của hiệu suất của mỗi người, bệnh nhân và các chuyên gia. Ý kiến về màu sắc của răng có một cực kỳ tính chất chủ quan và rất khác nhau giữa chuyên gia này và chuyên gia khác. Đa số bệnh nhân tự nhận thấy màu răng của họ tối hơn so với thực tế. Bệnh nhân nữ thường mong muốn có hàm răng sáng màu hơn\cite{LindaGreenwall2019}.\par
\quad Đánh giá nhu cầu làm trắng răng dựa trên 4 yếu tố: sự đổi màu răng, vị trí đổi màu răng, sự ảnh hưởng của màu sắc răng tới cuộc sống người bệnh, và nhu cầu mong muốn làm trắng răng của người bệnh.\par
\quad Bảng 3.2.1 cho thấy sự đổi màu răng ở các đối tượng nghiên cứu là như nhau ở hai giới nam và nữ. Tỷ lệ đổi màu vừa phải cao nhất (90,7\%), tỷ lệ đổi màu nghiêm trọng thấp hơn (4,6\%).\par
\quad Bảng 3.2.2 cho thấy tỷ lệ phân bố vị trí đổi màu răng ở hai giới cũng có sự tương đồng. Phân bố ở ít răng hoặc một răng chiếm tỷ lệ cao nhất (52,3\%). Tỷ lệ phân bố đều ở các răng thấp hơn (40\%). Tỷ lệ phân bố nhiều vết đậm nhạt đồng đều là 3\%. Tỷ lệ phân bố trên các răng phía trước thấp nhất, chiếm 1,5\%.\par
\quad Bảng 3.2.3 cho thấy tỷ lệ của ảnh hưởng màu sắc răng tới người bệnh khiến người bệnh mất tự tin một chút là cao nhất, chiếm khoảng 61,5\%. Tỷ lệ người bệnh không mất tự tin chút nào thấp hơn, khoảng 30,8\%. Tỷ lệ người bệnh hoàn toàn tự tin là rất đẹp và mất tự tin là ngang nhau, khoảng 3\%. \par
\quad Bảng 3.2.5 cho thấy tỷ lệ muốn răng được trắng hơn ở cả nam và nữ đều cao nhất (53,8\%), đặc biệt là nữ (74,3\% trong tổng nữ). Tỷ lệ đối tượng hiện tại đang không quan tâm và đợi làm trắng sau thấp hơn (khoảng 20\%). Tỷ lệ rất muốn được trắng hơn là thấp nhất, chỉ có 3,1\% và đều ở nữ giới.\par
\quad Trong nghiên cứu của Renata Pedrosa Guimarães (2021) về câu hỏi "Bạn có nghĩ rằng bạn cần phải làm trắng răng?"\cite{Renata2021} Tỷ lệ trả lời khẳng định là: nữ cao hơn nhiều so với nam (p=0,001). Các câu trả lời là khẳng định đối với hầu hết sinh viên trong tất cả các khóa học, ngoại trừ những người trong các khóa học y khoa với sự nhấn mạnh về dinh dưỡng học sinh (70,0\%) và học thể dục (63,6\%)\cite{LindaGreenwall2019}.\par
\quad Sự khác biệt giữa hai giới về mong muốn làm trắng răng ở trên cho thấy nữ giới thực sự rất quan tâm tới màu sắc răng và có nhu cầu khá cao về làm trắng răng.\par
\quad Bảng 3.2.4 cho thấy tỷ lệ nhu cầu làm trắng răng mức trung bình là cao nhất ở cả hai giới nam và nữ, chiếm khoảng 66,2\%. Tỷ lệ nhu cầu làm trắng răng mức mong muốn thấp hơn, chiếm 18,5\%. Tỷ lệ nhu cầu làm trắng răng mức khuyến cáo là 7,7\%, mức có thể là 4,6\% và ở mức cao là 3\%.\par
\quad Có thể nhấn mạnh rằng: hơn một nửa (53,9\%) câu trả lời là tẩy trắng răng; có một mối liên quan đáng kể giữa chỉ định của làm trắng với chuyên ngành và thời gian kể từ khi ra trường. Tỷ lệ các dấu hiệu tích cực cao hơn trong số bác sỹ thuộc các chuyên khoa: Răng Hàm Mặt, Răng Hàm Mặt + Chỉnh Nha và Răng Hàm Mặt + Cấy Ghép Implant, và thấp hơn giữa các chuyên khoa Răng Hàm Mặt + Nội Nha, Nội Nha, Nhi Khoa và Nha Chu. Cao nhất phần trăm phản hồi có chỉ định tẩy trắng xảy ra ở những người có hơn 20 năm tốt nghiệp (65,3\%) và thấp nhất là nhóm từ 10 đến 19 tuổi (35,6\%)\cite{LindaGreenwall2019}.\par
\quad Từ những kết quả trên, có thể nhận thấy rằng nhu cầu làm trắng răng được đánh giá dựa trên nhiều yếu tố để có thể đưa ra được kết luận phù hợp, đồng thời hài lòng cả bác sĩ và người bệnh.\par
\quad Bảng 3.2.6 cho thấy tỷ lệ màu sắc mong muốn của người bệnh ở màu 1M1 ở nữ là cao nhất, chiếm 36,4\% tổng số. Tỷ lệ màu sắc mong muốn của nữ giới ở màu 0M3 và 1M2 là như nhau (13,6\%). Tỷ lệ màu sắc mong muốn của nam giới ở màu 2M1 và 1M2 là cao nhất (7.7\%). Tỷ lệ màu sắc mong muốn ở nhóm 0 ở nữ giới cao hơn nam giới (ở nữ: 29,5\%; ở nam: 19\%).\par
\quad Từ đó, có thể thấy rằng nhu cầu làm trắng răng theo đánh giá của bác sĩ và mong muốn của người bệnh có một sự liên quan mật thiết, với nhu cầu thẩm mỹ của nữ giới là rất cao. Tuy nhiên bên cạnh đó có thể thấy được hiện nay nam giới cũng có nhu cầu nhất định liên quan đến thẩm mỹ.


