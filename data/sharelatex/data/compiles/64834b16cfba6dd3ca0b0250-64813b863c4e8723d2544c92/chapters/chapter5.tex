\chapter*{KẾT LUẬN}

\vspace{5pt}
\qquad Từ kết quả nghiên cứu đặc điểm màu sắc răng theo bảng màu Vita 3D Master và đánh giá nhu cầu làm trắng răng của người bệnh chúng tôi rút ra các kết luận sau:\par

\textbf{
\section*{1. Xác định đặc điểm màu sắc răng theo bảng màu Vita 3D Master}}
\quad Trong 65 người bệnh lựa chọn ngẫu nhiên, tỷ lệ nữ cao hơn nam khoảng hơn 2 lần. Độ tuổi của đối tượng nghiên cứu là 20-24 tuổi.\par
\quad Màu sắc răng: nhóm 1, nhóm 2 và nhóm 3 ở hàm trên và nhóm 2, nhóm 3 và nhóm 4 ở hàm dưới. \par 
\quad Tỷ lệ phân bố giá trị màu sắc răng hàm trên ở nhóm 3 là cao nhất ở cả 3 nhóm răng (nhóm răng cửa (50,8\%), nhóm răng nanh (53,8\%), nhóm răng hàm nhỏ(53,8\%)). Tỷ lệ màu sắc răng ở nhóm 2 xếp sau với nhóm răng cửa khoảng 43\%, nhóm răng nanh khoảng 40\% và nhóm răng hàm nhỏ khoảng 41,5\%. Nhóm 1 chỉ chiếm tỷ lệ rất nhỏ là 1,5\% và không có đối tượng nào có màu sắc theo nhóm 5.\par
\quad Tỷ lệ phân bố giá trị màu sắc răng hàm dưới ở nhóm 3 là cao nhất ở cả 3 nhóm răng (nhóm răng cửa (68,8\%), nhóm răng nanh (61\%), nhóm răng hàm nhỏ (70,3\%)). Tỷ lệ màu sắc răng ở nhóm 2 xếp sau với nhóm răng cửa khoảng 26,6\%, nhóm răng nanh khoảng 25\% và nhóm răng hàm nhỏ khoảng 25\%. Nhóm 1 và nhóm 5 không có đối tượng nào.\par
\quad Tỷ lệ màu sắc răng cao nhất ở các đối tượng là màu 3M2, tiếp theo là 3M3 và 3R2.5. Tỷ lệ màu sắc răng thấp nhất là màu 4M1,4R2.5 và 1M2.\par 
\quad Màu sắc răng giữa các nhóm răng cửa, răng nanh và hàm nhỏ là không đồng nhất, màu đậm nhất là nhóm răng nanh. Đa số màu sắc răng ở màu trung tính, một số răng có độ đỏ và độ vàng, chủ yếu ở nhóm răng nanh.\par 
\quad Bề mặt răng đa số nhẵn bóng, không nứt, có tính chất đồng nhất và không sâu răng. 
\vspace{-5pt}
\textbf{
\section*{2. Đánh giá nhu cầu làm trắng răng ở sinh viên trường Đại học Y Dược}}
\quad Tỷ lệ của ảnh hưởng màu sắc răng tới người bệnh khiến người bệnh mất tự tin một chút là cao nhất, chiếm khoảng 61,5\%. Tỷ lệ người bệnh không mất tự tin chút nào thấp hơn, khoảng 30,8\%. Tỷ lệ người bệnh hoàn toàn tự tin là rất đẹp và mất tự tin là ngang nhau, khoảng 3\%.\par
\quad Tỷ lệ muốn răng được trắng hơn ở cả nam và nữ đều cao nhất (53,8\%), đặc biệt là nữ (74,3\% trong tổng nữ). Tỷ lệ đối tượng hiện tại đang không quan tâm và đợi làm trắng sau thấp hơn (khoảng 20\%). Tỷ lệ rất muốn được trắng hơn là thấp nhất, chỉ có 3,1\% và đều ở nữ giới.\par
\quad Tỷ lệ nhu cầu làm trắng răng mức trung bình là cao nhất ở cả hai giới nam và nữ, chiếm khoảng 66,2\%. Tỷ lệ nhu cầu làm trắng răng mức mong muốn thấp hơn, chiếm 18,5\%. Tỷ lệ nhu cầu làm trắng răng mức khuyến cáo là 7,7\%, mức có thể là 4,6\% và ở mức cao là 3\%.\par
\quad Tỷ lệ màu sắc mong muốn của người bệnh ở màu 1M1 ở nữ là cao nhất, chiếm 36,4\% tổng số. Tỷ lệ màu sắc mong muốn của nữ giới ở màu 0M3 và 1M2 là như nhau (13,6\%). Tỷ lệ màu sắc mong muốn của nam giới ở màu 2M1 và 1M2 là cao nhất (7.7\%). Tỷ lệ màu sắc mong muốn ở nhóm 0 ở nữ giới cao hơn nam giới (ở nữ: 29,5\%; ở nam: 19\%).

