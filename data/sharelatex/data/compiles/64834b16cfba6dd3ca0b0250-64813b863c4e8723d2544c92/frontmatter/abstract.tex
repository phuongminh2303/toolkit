\abstractpage{
\par
\quad Trong xã hội hiện đại, thẩm mỹ và ngoại hình, đặc biệt là khuôn mặt, đóng vai trò vô cùng quan trọng trong việc tạo dựng ấn tượng ban đầu và tăng khả năng tự tin của các cá nhân. Cảm quan thị giác của con người thường được sử dụng để đánh giá và tạo ra những đánh giá ban đầu về một người, một nơi hoặc một vật. Vì vậy, có một diện mạo hài hoà và hấp dẫn có thể tạo ra một ấn tượng mạnh mẽ từ ban đầu. Thẩm mỹ không chỉ ảnh hưởng đến sự tự tin và hài lòng về bản thân, mà còn có thể tác động đến khả năng giao tiếp và tương tác xã hội. Ngoài ra, thẩm mỹ cũng là yếu tố quan trọng trong nghệ thuật và thiết kế và được thể hiện trong nhiều lĩnh vực khác nhau.\par
\quad Hàm răng đóng vai trò quan trọng trong thẩm mỹ của khuôn mặt. Nó tạo nên khung xương cơ bản và hỗ trợ các cơ mặt, giữ cho hình dạng và độ căng của môi và khuôn mặt. Răng và hàm răng khoẻ mạnh giúp duy trì một khuôn mặt đầy đặn. Trong số các yếu tố thẩm mỹ của khuôn mặt, màu sắc răng đóng vai trò quan trọng trong việc tạo nên một nụ cười hấp dẫn và có thể ảnh hưởng đáng kể đến cảm giác tự tin và sự hài lòng về bản thân. Màu sắc răng không chỉ tạo nên một nụ cười đẹp, mà nó còn mang theo thông điệp về sức khoẻ và chăm sóc cá nhân. Răng trắng và đều màu thường được coi là dấu hiệu sức khoẻ và làm tăng sự tự tin khi cười. Ngược lại, răng có màu sậm, vết ố vàng, hoặc màu không đồng đều có thể làm giảm sự tự tin và gây ngại khi cười. \par
\quad Có nhiều nghiên cứu đã chứng minh tầm quan trọng của màu sắc răng trong thẩm mỹ và tự tin. Một nghiên cứu được công bố trong tạp chí "Journal of Dentistry" đã khảo sát ý kiến của người tham gia về thẩm mỹ của màu sắc răng. Màu sắc răng có thể ảnh hưởng đến việc đánh giá vẻ ngoài và sự tự tin của người khác. Nghiên cứu cũng chỉ ra mối liên quan giữa màu sắc răng và sự cảm nhận về tuổi tác. Một khảo sát ý kiến của bác sĩ nha khoa và người dùng cũng xác nhận vai trò quan trọng của màu sắc răng trong thẩm mỹ và sự hài lòng của bệnh nhân. \par
\quad Hiện nay ở Việt Nam, tỷ lệ nhiễm sắc khá cao trong cộng đồng. Theo Đỗ Quang Trung và cộng sự (CS) (2010), có 86,9\% người dân có răng bị nhiễm sắc và 89,97\% người dân mong muốn có hàm răng trắng đẹp\cite{NguyenThiChau}. Màu sắc của răng được quyết định bởi yếu tố màu sắc bên trong của răng và cả nhiễm màu do các yếu tố bên ngoài gây ra. Do vậy, với tầm quan trọng này, nhu cầu làm trắng răng ngày càng được quan tâm và phát triển.\par
\quad Có nhiều nghiên cứu về màu sắc răng ở Việt Nam, tuy nhiên, hiện vẫn thiếu nghiên cứu đi sâu đánh giá màu sắc răng và nhu cầu làm trắng răng đặc biệt là ở sinh viên đại học ở Việt Nam. Vì vậy, chúng tôi tiến hành luận án: \textbf{“Đặc điểm màu răng theo bảng màu Vita 3D và xác định nhu cầu làm trắng răng ở sinh viên trường Đại học Y Dược - Đại học Quốc gia Hà Nội”} với hai mục tiêu sau: \par
\begin{enumerate}
\setlength{\itemsep}{0pt}
    \item \textbf{Xác định đặc điểm màu răng theo bảng Vita 3D ở sinh viên trường Đại học Y Dược - Đại học Quốc Gia Hà Nội năm 2023.}
    \item \textbf{Xác định nhu cầu làm trắng răng ở đối tượng trên.}
\end{enumerate}
}